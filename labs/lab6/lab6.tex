\documentclass[letterpaper, 10pt, DIV=13]{scrartcl}
\usepackage[T1]{fontenc}
\usepackage[english]{babel}
\usepackage{amsmath, amsfonts, amsthm, xfrac}
\usepackage{listings}
\usepackage{color}
\usepackage{longtable}
\usepackage{qtree}

\numberwithin{equation}{section}
\numberwithin{figure}{section}
\numberwithin{table}{section}

\usepackage{sectsty}
\allsectionsfont{\normalfont\scshape} % Make all section titles in default font and small caps.

\usepackage{fancyhdr} % Custom headers and footers
\pagestyle{fancyplain} % Makes all pages in the document conform to the custom headers and footers

\fancyhead{} % No page header - if you want one, create it in the same way as the footers below
\fancyfoot[L]{} % Empty left footer
\fancyfoot[C]{} % Empty center footer
\fancyfoot[R]{\thepage} % Page numbering for right footer

\renewcommand{\headrulewidth}{0pt} % Remove header underlines
\renewcommand{\footrulewidth}{0pt} % Remove footer underlines
\setlength{\headheight}{13.6pt} % Customize the height of the header

\setlength\parindent{0pt}
\pagenumbering{gobble}

\title {
	\normalfont
	\huge{Lab 6} \\
	\vspace{10pt}
	\large{CMPT 432 - Spring 2023 | Dr. Labouseur}
}

\author{\normalfont Josh Seligman | joshua.seligman1@marist.edu}

\pagenumbering{arabic}

\definecolor{mygreen}{rgb}{0,0.6,0}
\definecolor{mygray}{rgb}{0.5,0.5,0.5}
\definecolor{mymauve}{rgb}{0.58,0,0.82}
\lstset{
  backgroundcolor=\color{white},   % choose the background color
  basicstyle=\footnotesize,        % size of fonts used for the code
  breaklines=true,                 % automatic line breaking only at whitespace
  captionpos=b,                    % sets the caption-position to bottom
  commentstyle=\color{mygreen},    % comment style
  escapeinside={\%*}{*},          % if you want to add LaTeX within your code
  keywordstyle=\color{blue},       % keyword style
  stringstyle=\color{mymauve},     % string literal style
}

\begin{document}
\maketitle

\section{Crafting a Compiler}
The two data structures most commonly used to implement symbol
tables in production compilers are binary search trees and hash tables.
What are the advantages and disadvantages of using each of these data
structures for symbol tables?
\\ \\
The primary advantage of using a binary search tree for a symbol table is that it 
is only bounded by the size of memory, assuming the nodes are connected through 
objects and pointers rather than a matrix or adjacency list. The downside for binary
search trees is that they have a lookup and insertion times of $O(log_2n)$, where
$n$ is the number of symbols in the tree. On the other hand, hash tables are extremely
efficient with lookup times and insertion times with a time complexity of
$O(1) + \alpha$. This $\alpha$ is the extra time it may take to probe or iterate
through a chain in the case of collisions. Hash tables with chaining, similar to
trees, are not bound by a fixed allocation because of the use of a linked list.
However, hash tables that use probing to handle collisions are restricted to the
length of the allocated array, which may be too small to store all variables in a given scope.

\section{Dragon}
\end{document}
