\documentclass[letterpaper, 10pt, DIV=13]{scrartcl}
\usepackage[T1]{fontenc}
\usepackage[english]{babel}
\usepackage{amsmath, amsfonts, amsthm, xfrac}
\usepackage{listings}
\usepackage{color}
\usepackage{longtable}
\usepackage{qtree}

\numberwithin{equation}{section}
\numberwithin{figure}{section}
\numberwithin{table}{section}

\usepackage{sectsty}
\allsectionsfont{\normalfont\scshape} % Make all section titles in default font and small caps.

\usepackage{fancyhdr} % Custom headers and footers
\pagestyle{fancyplain} % Makes all pages in the document conform to the custom headers and footers

\fancyhead{} % No page header - if you want one, create it in the same way as the footers below
\fancyfoot[L]{} % Empty left footer
\fancyfoot[C]{} % Empty center footer
\fancyfoot[R]{\thepage} % Page numbering for right footer

\renewcommand{\headrulewidth}{0pt} % Remove header underlines
\renewcommand{\footrulewidth}{0pt} % Remove footer underlines
\setlength{\headheight}{13.6pt} % Customize the height of the header

\setlength\parindent{0pt}
\pagenumbering{gobble}

\title {
	\normalfont
	\huge{Lab 4} \\
	\vspace{10pt}
	\large{CMPT 432 - Spring 2023 | Dr. Labouseur}
}

\author{\normalfont Josh Seligman | joshua.seligman1@marist.edu}

\pagenumbering{arabic}

\definecolor{mygreen}{rgb}{0,0.6,0}
\definecolor{mygray}{rgb}{0.5,0.5,0.5}
\definecolor{mymauve}{rgb}{0.58,0,0.82}
\lstset{
  backgroundcolor=\color{white},   % choose the background color
  basicstyle=\footnotesize,        % size of fonts used for the code
  breaklines=true,                 % automatic line breaking only at whitespace
  captionpos=b,                    % sets the caption-position to bottom
  commentstyle=\color{mygreen},    % comment style
  escapeinside={\%*}{*},          % if you want to add LaTeX within your code
  keywordstyle=\color{blue},       % keyword style
  stringstyle=\color{mymauve},     % string literal style
}

\begin{document}
\maketitle

\section{Crafting a Compiler}
\subsection{Exercise 4.9}
Compute the First and Follow sets for the nonterminals of the following grammar:
\begin{enumerate}
    \item S -> a S e
    \item ~~~~~| B
    \item B -> b B e
    \item ~~~~~| C
    \item C -> c C e
    \item ~~~~~| d
\end{enumerate}

S: first set = \{a, b, c, d\}, follow set = \{\$, e\} \\
B: first set = \{b, c, d\}, follow set = \{e, \$\} \\
C: first set = \{c, d\}, follow set = \{e, \$\} \\

\begin{tabular}{|c|c|c|c|c|c|}
    \hline
    Nonterminal & a & b & c & d & e \\
    \hline
    S & \#1 & \#2 & \#2 & \#2 & \\
    \hline
    B & & \#3 & \#4 & \#4 & \\
    \hline
    C & & & & \#5 & \#6 \\
    \hline
\end{tabular}

\subsection{Exercise 5.10}
Show 2 distinct parse trees that can be constructed for: if expr then if expr then other else other
\begin{enumerate}
    \item S -> Stmt \$
    \item Stmt --> if expr then Stmt else Stmt
    \item ~~~~~~~~~~| if expr then Stmt
    \item ~~~~~~~~~~| other
\end{enumerate}

The 2 trees are below. The important difference between them is which if statement does the else pair up with. In the first tree, the else pairs with the most recent if and the second subtree pairs the else with the outer if.

\Tree [.Start
        [.Stmt 
            [.{if~expr~then} ]
            [.Stmt 
                [.{if~expr~then} ]
                [.Stmt
                    [.other ]
                ]
                [.else ]
                [.Stmt
                    [.other ]
                ]
            ]
        ] 
        [.\$ ]
      ]
\\ \\
\Tree [.Start
        [.Stmt 
            [.{if~expr~then} ]
            [.Stmt 
                [.{if~expr~then} ]
                [.Stmt
                    [.other ]
                ]
            ]
            [.else ]
            [.Stmt
                [.other ]
            ]
        ] 
        [.\$ ]
      ]
\section{Dragon}
Compute first and follow sets for the following grammar:
\begin{enumerate}
    \item S --> S S +
    \item ~~~~~~| S S *
    \item ~~~~~~| a
\end{enumerate}

First(S) = \{a\}
Follow(S) = \{+, *, \$, a\}

\end{document}
