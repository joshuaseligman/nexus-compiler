\documentclass[letterpaper, 10pt, DIV=13]{scrartcl}
\usepackage[T1]{fontenc}
\usepackage[english]{babel}
\usepackage{amsmath, amsfonts, amsthm, xfrac}
\usepackage{listings}
\usepackage{color}
\usepackage{longtable}
\usepackage{qtree}

\numberwithin{equation}{section}
\numberwithin{figure}{section}
\numberwithin{table}{section}

\usepackage{sectsty}
\allsectionsfont{\normalfont\scshape} % Make all section titles in default font and small caps.

\usepackage{fancyhdr} % Custom headers and footers
\pagestyle{fancyplain} % Makes all pages in the document conform to the custom headers and footers

\fancyhead{} % No page header - if you want one, create it in the same way as the footers below
\fancyfoot[L]{} % Empty left footer
\fancyfoot[C]{} % Empty center footer
\fancyfoot[R]{\thepage} % Page numbering for right footer

\renewcommand{\headrulewidth}{0pt} % Remove header underlines
\renewcommand{\footrulewidth}{0pt} % Remove footer underlines
\setlength{\headheight}{13.6pt} % Customize the height of the header

\setlength\parindent{0pt}
\pagenumbering{gobble}

\title {
	\normalfont
	\huge{Lab 3} \\
	\vspace{10pt}
	\large{CMPT 432 - Spring 2023 | Dr. Labouseur}
}

\author{\normalfont Josh Seligman | joshua.seligman1@marist.edu}

\pagenumbering{arabic}

\definecolor{mygreen}{rgb}{0,0.6,0}
\definecolor{mygray}{rgb}{0.5,0.5,0.5}
\definecolor{mymauve}{rgb}{0.58,0,0.82}
\lstset{
  backgroundcolor=\color{white},   % choose the background color
  basicstyle=\footnotesize,        % size of fonts used for the code
  breaklines=true,                 % automatic line breaking only at whitespace
  captionpos=b,                    % sets the caption-position to bottom
  commentstyle=\color{mygreen},    % comment style
  escapeinside={\%*}{*},          % if you want to add LaTeX within your code
  keywordstyle=\color{blue},       % keyword style
  stringstyle=\color{mymauve},     % string literal style
}

\begin{document}
\maketitle

\section{Crafting a Compiler}
\subsection{Exercise 4.7}
Given the following grammar:
\begin{enumerate}
    \item Start --> E \$
    \item E --> T plus E
    \item ~~~~~~| T
    \item T --> T times F
    \item ~~~~~~| F
    \item F --> ( E )
    \item ~~~~~~| num
\end{enumerate}

\begin{enumerate}
    \item Show the leftmost derivation of the following string: num plus num times num plus num \$ \\ \\
    S. Start \\
    1. E \$ \\
    2. T plus E \$ \\
    5. F plus E \$ \\
    7. num plus E \$ \\
    2. num plus T plus E \$ \\
    4. num plus T times F plus E \$ \\
    5. num plus F times F plus E \$ \\
    7. num plus num times F plus E \$ \\
    7. num plus num times num plus E \$ \\
    3. num plus num times num plus T \$ \\
    5. num plus num times num plus F \$ \\
    7. num plus num times num plus num \$ \\

    \Tree [.Start
            [.E
                [.T
                    [.F 
                        [.num ]
                    ]
                ]
                [.plus ]
                [.E
                    [.T 
                        [.T 
                            [.F 
                                [.num ]
                            ]
                        ]
                        [.times ]
                        [.F 
                            [.num ]
                        ]
                    ]
                    [.plus ]
                    [.E 
                        [.T 
                            [.F 
                                [.num ]
                            ]
                        ]
                    ]
                ]
            ]
            [.\$ ]
          ]


    \item Show the rightmost derivation of the following string: num times num plus num times num \$ \\ \\
    S. Start \\
    1. E \$ \\
    2. T plus E \$ \\
    3. T plus T \$ \\
    4. T plus T times F \$ \\
    7. T plus T times num \$ \\
    5. T plus F times num \$ \\
    7. T plus num times num \$ \\
    4. T times F plus num times num \$ \\
    7. T times num plus num times num \$ \\
    5. F times num plus num times num \$ \\
    7. num times num plus num times num \$ \\

    \Tree [.Start
            [.E 
                [.T 
                    [.T 
                        [.F 
                            [.num ]
                        ]
                    ]
                    [.times ]
                    [.F 
                        [.num ]
                    ]
                ]
                [.plus ]
                [.E 
                    [.T 
                        [.T 
                            [.F 
                                [.num ]
                            ]
                        ]
                        [.times ]
                        [.F 
                            [.num ]
                        ]
                    ]
                ]
            ]
            [.\$ ]
          ]

    \item Describe how this grammar structures expressions, in terms of the precedence and left- or right- associativity of operators. \\ \\
    This grammar structures expressions by forcing the higher precedence operators to be placed lower in the parse tree. The grammar creates this precedence by having lower precedence operators go through the productions for higher-precedence operators before reaching a terminal, which will enable the higher-precedence operator to be evaluated first because it is lower in the CST. Right-most operators also get precedence over left-most operators because production rule 2 says that E --> T plus E, which means that if there is another plus, the E on the right would have to expand unless the T goes down eventually to production rule 6 and adds the parentheses.
\end{enumerate}

\subsection{Exercise 5.2c}
Given the following LL(1) grammar, write a recursive descent parser.
\begin{enumerate}
    \item Start --> Value \$ \\
    \item Value --> num \\
    \item ~~~~~~~~~~~| lparen Expr rparen \\
    \item Expr --> plus Value Value \\
    \item ~~~~~~~~~~| prod Values \\
    \item Values --> Value Values \\
    \item ~~~~~~~~~~~~~| $\lambda$
\end{enumerate}

\begin{lstlisting}[frame=single]
func parseStart() {
    parseValue()
    match($)
}

func parseValue() {
    if peek() == lparen {
        match(lparen)
        parseExpr()
        match(rparen)
    } else {
        match(num)
    }
}

func parseExpr() {
    if peek() == plus {
        match(plus)
        parseValue()
        parseValue()
    } else {
        match(prod)
        parseValues()
    }
}

func parseValues() {
    // Check for starter tokens for Value
    if peek() == num || peek() == lparen {
        parseValue()
        parseValues()
    } else {
        // Nothing to do here, epsilon condition
    }
}
\end{lstlisting}

\section{Dragon}
\subsection{Exercise 4.2.1}
Given the following CFG and the string aa+a*:
\begin{enumerate}
    \item S --> S S +
    \item ~~~~~~| S S *
    \item ~~~~~~| a
\end{enumerate}

\begin{enumerate}
    \item Give a leftmost derivation of the string. \\ \\
    Start. S \\
    2. S S * \\
    1. S S + S * \\
    3. a S + S * \\
    3. a a + S * \\
    3. a a + a * \\

    \item Give a rightmost derivation of the string. \\ \\
    Start. S \\
    2. S S * \\
    3. S a * \\
    1. S S + a * \\
    3. S a + a * \\
    3. a a + a * \\

    \item Give a parse tree for the string. \\ \\
    \Tree [.S
            [.S
                [.S
                    [.a ]
                ]
                [.S
                    [.a ]
                ]
                [.+ ]
            ]
            [.S
                [.a ]
            ]
            [.* ]
          ]
\end{enumerate}

\end{document}